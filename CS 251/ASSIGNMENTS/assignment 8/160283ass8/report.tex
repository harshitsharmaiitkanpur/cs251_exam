\documentclass[a4paper, 10pt]{report}
\usepackage[utf8]{inputenc}
\usepackage{graphicx}
\usepackage{subcaption}
\usepackage{float}
\usepackage{geometry}
 \geometry{
 a4paper,
 total={170mm,257mm},
 left=22mm,
 top=22mm,
 }

\title{Latex Report on Time taken.\\CS251}
\author{Harshit Sharma\\160283}

\begin{document}

\maketitle

\chapter{Scatter plots}

\begin{figure}
\centering
\includegraphics[width=2\textwidth]{scatter1}
 \caption{Scatter plot using 1 thread}
 \label{fig:cc_get}
\end{figure}

\begin{figure}
\centering
\includegraphics[width=2\textwidth]{scatter2}
 \caption{Scatter plot using 2 threads}
 \label{fig:cc_get}
\end{figure}

\begin{figure}
\centering
\includegraphics[width=2\textwidth]{scatter3}
 \caption{Scatter plot using 4 threads}
 \label{fig:cc_get}
\end{figure}

\begin{figure}
\centering
\includegraphics[width=2\textwidth]{scatter4}
 \caption{Scatter plot using 8 threads}
 \label{fig:cc_get}
\end{figure}

\begin{figure}
\centering
\includegraphics[width=2\textwidth]{scatter5}
 \caption{Scatter plot using 16 threads}
 \label{fig:cc_get}
\end{figure}

\chapter{Line plot}

\begin{figure}
\centering
\includegraphics[width=2\textwidth]{line plot3(b)}
 \caption{Line plot showing average execution time taken versus number of elements using 1,2,4,8 or 16 threads}
\end{figure}



\end{document}
